\documentclass{beamer}
%\documentclass[handout]{beamer}
% This file requires a \documentclass{beamer} before and a \date{..} after

\usepackage[utf8]{inputenc}
\usepackage{graphicx}
\usepackage{ifthen}
\usepackage{moodle}
\usepackage{xcolor}
\usepackage{listings}
\usepackage{makecell}
\usepackage{tikz}
\usepackage{url}
\usepackage{mathtools}
\usepackage{ulem}
\usepackage{mathcomp}
\usepackage[italian]{babel}
\usepackage{amsmath,amsthm,amssymb,amsfonts}
\usepackage{setspace}
\usepackage{tikz}
\usepackage{soul}
\usepackage[shortlabels]{enumitem}

% Global packages
\usepackage{tikzit}

\input{style.tikzstyles}
\usetheme[secheader]{Boadilla}
\usecolortheme{dolphin}

% Main color
\definecolor{grin}{HTML}{53bbc2}
\definecolor{redd}{HTML}{b82200}
\definecolor{uait}{HTML}{ffffff}
\definecolor{blec}{HTML}{000000}


% Colors
\setbeamercolor*{block body}{bg=normal text.bg!93!grin}
\setbeamercolor*{block title}{bg=grin!25!uait}

\setbeamercolor{normal text}{fg=blec!65!grin,bg=uait}

\setbeamercolor{section in toc}{fg=grin}
\setbeamercolor{subsection in toc}{fg=grin!90!uait}

\setbeamercolor{frametitle}{fg=grin,bg=grin!20}

\setbeamercolor*{palette primary}{use=structure,fg=blec,bg=grin!40!uait}
\setbeamercolor*{palette secondary}{use=structure,fg=uait,bg=grin!60!uait}
\setbeamercolor*{palette tertiary}{use=structure,fg=uait,bg=grin!90!uait}
\setbeamercolor*{palette quaternary}{fg=uait,bg=blec}

\setbeamercolor*{sidebar}{use=structure,bg=grin}

\setbeamercolor*{palette sidebar primary}{use=structure,fg=grin!10}
\setbeamercolor*{palette sidebar secondary}{fg=grin!10!uait}
\setbeamercolor*{palette sidebar tertiary}{use=structure,fg=grin!50}
\setbeamercolor*{palette sidebar quaternary}{fg=grin!5!uait}

% Section balls
\setbeamerfont{section number projected}{family=\rmfamily,series=\bfseries,size=\normalsize}
\setbeamercolor{section number projected}{bg=grin!70!uait,fg=uait}

% Itemize balls
\setbeamercolor{item projected}{bg=grin!70!uait}
\setbeamercolor{subitem projected}{bg=grin!90!uait}

% Title
\setbeamercolor*{titlelike}{use=structure,fg=grin}
%lst colors

\definecolor{commentgreen}{RGB}{162, 249, 103}
\definecolor{eminence}{RGB}{108,48,130}
\definecolor{weborange}{RGB}{255,165,0}
\definecolor{frenchplum}{RGB}{254,215,0}
\definecolor{goodred}{RGB}{233, 38, 95}
\definecolor{washedyellow}{RGB}{223, 206, 94}
\definecolor{monoblue}{RGB}{79, 201, 232}

% blue code background
\lstset{
    language=c++,
    frame=single,
    framerule=0pt,
    backgroundcolor=\color{grin!70!blec},
   % basicstyle=\ttfamily\color{white},
    xleftmargin=4.3pt,
    xrightmargin=4.3pt,
    commentstyle=\color{commentgreen},
    stringstyle=\color{washedyellow},
    basicstyle=\small\ttfamily\color{white}, % basic font setting
    keywordstyle = \color{eminence},
    keywordstyle = [2]{\color{weborange}},
    keywordstyle = [3]{\color{monoblue}},
    otherkeywords = {>,<,;,.,-,!,=,~,memcpy,send_request_cgi,encoded,gsub},
    morekeywords = [2]{>,<,;,.,-,!,=,~},
    morekeywords = [3]{memcpy,send_request_cgi,encoded,gsub},
    %emph={int,char,double,float,unsigned,void,bool},
    %emphstyle={\color{blue}},
    escapechar=\&
}


% transparent background in slides
\setbeamercovered{transparent}
\setbeameroption{show notes}

\newcounter{numeroes}[section]
\newcounter{requestcounter}

\DeclareMathAlphabet{\pazocal}{OMS}{zplm}{m}{n}

%\renewenvironment{proof}{\noindent\textbf{Soluzione:}}{}
\newtheorem{exercise}{Esercizio}

\addto\captionsitalian{\renewcommand\proofname{Soluzione}}
\renewcommand*{\proofname}{Solution}
\renewcommand\qedsymbol{$\blacksquare$}

\newcommand{\mean}[1]{$\mathbb{E}[#1]$}
\newcommand{\varr}[1]{$\mathbb{V}ar[#1]$}
%\renewcommand\qedsymbol{$\square$}

\setlength{\jot}{10pt}

% tikz round photos
\newcommand{\roundpic}[4][]{
	\tikz\node [circle, minimum width = #2,
		path picture = {
				\node [#1] at (path picture bounding box.center) {
					\includegraphics[width=#3]{#4}};
			}] {};}

% Header slide
\title[Tutorato Probabilità]
{Tutorato di\\Calcolo delle Probabilità}

\author[DISI]{%
	\texorpdfstring{%
		\begin{columns}
			\column{.50\linewidth}
			\centering
			Camilla Righetti
			\column{.50\linewidth}
			\centering
			Matteo Franzil
		\end{columns}
	}{DISI}
}

\institute[UniTN]{
	\texorpdfstring{%
		\includegraphics[width=.6\textwidth]{drawable/logos/logo-disi.png}
	}{University of Trento}\\
	\smallskip
	%\hphantom{...}\\
	Corso di Laurea in Ingegneria Informatica, delle Comunicazioni ed Elettronica
}

\date[24/05/2022]{24 maggio 2022}

\begin{document}

\frame{\titlepage}

\begin{frame}
	\frametitle{Esercizio d'esempio 1}
	\begin{exercise}[20210524-1]
		Consideriamo la seguente funzione:

		\[
			f(x) = 4 x^3
		\]

		Questa funzione è la \textit{pdf} della variabile aleatoria $X$ definita nel dominio $x \in [0,1]$. Definiamo una seconda variabile aleatoria $Y$ con densità:

		\[
			y = 8 \sqrt{x}
		\]

		\begin{enumerate}[(a)]
			\item Ricavare la \textit{cdf} della variabile aleatoria $X$.
			\item Ricavare la \textit{pdf} della variabile aleatoria $Y$.
			\item Ricavare la \textit{cdf} della variabile aleatoria $Y$.
		\end{enumerate}
	\end{exercise}
\end{frame}

\begin{frame}[fragile]
	\begin{proof}\renewcommand{\qedsymbol}{$\square$}
		\begin{enumerate}[(a)]
			\item Ricavare la \textit{cdf} è facile: integriamo la funzione dal punto $0$ fino al punto $1$.
			      \[
				      \int_0^1 f(x) ,dx = \int_0^1 4x^3 ,dx = x^4
			      \]
		\end{enumerate}
	\end{proof}
\end{frame}

\begin{frame}[fragile]
	\begin{proof}\renewcommand{\qedsymbol}{$\longrightarrow$}
		\begin{enumerate}[(b)]
			\item Per ricavare la \textit{pdf} di Y, dobbiamo invertire la funzione di trasformazione $g(x) = y$. Allora abbiamo che
			      \[
				      g^{-1}(y) = \big(\frac{x}{8}\big)^2
			      \]
			      e
			      \[
				      \frac{dg^{-1}(y)}{dy} = \frac{1}{32} x
			      \]
		\end{enumerate}
	\end{proof}
\end{frame}

\begin{frame}[fragile]
	\begin{proof}\renewcommand{\qedsymbol}{$\blacksquare$}
		\begin{enumerate}[(b)]
			\item Allora ora possiamo procedere con la trasformazione vera e propria. Abbiamo dunque:
			      \[
				      f_y(y) = f_x(g^{-1}(y)) \cdot \frac{dg^{-1}(y)}{dy} = 4 \cdot ((\frac{x}{8})^2)^3 \cdot \frac{1}{32} x = \frac{1}{2097152} x^7
			      \]
			      Per trovare gli estremi, verifichiamo che la funzione passa l'asse $y$ in $x = 0$, mentre si interseca con l'inversa in $x = 8$. Verifichiamo effettivamente che l'integrale da 0 a 8 di questa funzione è pari a 1.
		\end{enumerate}
		\begin{enumerate}[(c)]
			\item L'ultimo punto lo si verifica trivialmente integrando la sopracitata funzione. Otteniamo $F(Y) = \frac{1}{16777216} x^8$
		\end{enumerate}
	\end{proof}
\end{frame}

\begin{frame}
	\frametitle{Esercizio d'esempio 2}
	\begin{exercise}[20210524-2]
		La quantità di gin tonic versata in un bicchiere al Mercolegin si comporta come una variabile aleatoria normale, con parametri $\mu = 19 \: ml$ e $\sigma^2 = 16 \: ml^2$.

		\begin{enumerate}[(a)]
			\item Quale è la probabilità che un bicchiere di gin tonic contenga almeno $25 \: ml$ di drink?
			\item Uno studente, terminata la seconda provetta di Calcolo delle probabilità, decide di festeggiare andando al Mercolegin per bere 3 bicchieri di gin tonic. Sapendo che ha bisogno di almeno $21 \: ml$ di gin tonic per divertirsi, ma sboccherebbe se ne bevesse più di $65 \: ml$, quale è la probabilità che si diverta senza sboccare?
		\end{enumerate}
	\end{exercise}
\end{frame}

\begin{frame}
	\frametitle{Esercizio d'esempio 2}
	\begin{exercise}[20210524-2]
		\begin{enumerate}[(c)]
			\item Per attirare più clientela, il Bar Domo il mercoledì successivo decide di cambiare bicchieri. Con questi nuovi bicchieri, la varianza è sempre $\sigma^2 = 16 \: ml^2$, ma la media è ignota. Degli studenti, dopo alcuni ``attenti'' calcoli, verificano che la probabilità che questi bicchieri contengano al massimo $25 \: ml$ di drink è 0,853. Quale è la media di questa nuova distribuzione?
		\end{enumerate}
	\end{exercise}
\end{frame}

\begin{frame}[fragile]
	\begin{proof}\renewcommand{\qedsymbol}{$\square$}
		\begin{enumerate}[(a)]
			\item $p = 1 - F(\frac{25 \: ml - 19 \: ml}{\sqrt{16 \: ml^2}}) = 1 - F(\frac{6}{4}) = 1 - F(1.5) = 1 - 0.933192 = 0.066808$
			\item Abbiamo una nuova variabile aleatoria con $\mu = 19 \cdot 3 \: ml$ e $\sigma^2 = 16 \cdot 3 \: ml^2$, quindi $\mu = 57 \: ml$ e $\sigma^2 = 48 \: ml^2$. Allora
			      \begin{equation*}
				      \begin{split}
					      P(21 \le X \le 65) & = (\frac{65 \: ml - 57 \: ml}{\sqrt{48 \: ml^2}}) - F(\frac{21 \: ml - 57 \: ml}{\sqrt{48 \: ml^2}}) \\
					      & = F(1.154701) - F(-5.196) \\
						  & = 0.8758
				      \end{split}
			      \end{equation*}
		\end{enumerate}
	\end{proof}
\end{frame}

\begin{frame}[fragile]
	\begin{proof}\renewcommand{\qedsymbol}{$\square$}
		\begin{enumerate}[(c)]
			\item Cerchiamo nella tavola $0.853$: otteniamo 1.05 (ovvero, $F(1.05) = 0.853$). Allora abbiamo $z = 1.05$. Possiamo usare la formula inversa della standardizzazione:
			      \[
				      z = 1.05 = \frac{25 \: ml - \mu}{\sqrt{16 \: ml^2}} \Leftrightarrow 25 -(1.05 \cdot 4) = \mu = 20.8 \: ml
			      \]
		\end{enumerate}
	\end{proof}
\end{frame}

\end{document}
