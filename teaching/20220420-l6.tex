\documentclass{beamer}
%\documentclass[handout]{beamer}
% This file requires a \documentclass{beamer} before and a \date{..} after

\usepackage[utf8]{inputenc}
\usepackage{graphicx}
\usepackage{ifthen}
\usepackage{moodle}
\usepackage{xcolor}
\usepackage{listings}
\usepackage{makecell}
\usepackage{tikz}
\usepackage{url}
\usepackage{mathtools}
\usepackage{ulem}
\usepackage{mathcomp}
\usepackage[italian]{babel}
\usepackage{amsmath,amsthm,amssymb,amsfonts}
\usepackage{setspace}
\usepackage{tikz}
\usepackage{soul}
\usepackage[shortlabels]{enumitem}

% Global packages
\usepackage{tikzit}

\input{style.tikzstyles}
\usetheme[secheader]{Boadilla}
\usecolortheme{dolphin}

% Main color
\definecolor{grin}{HTML}{53bbc2}
\definecolor{redd}{HTML}{b82200}
\definecolor{uait}{HTML}{ffffff}
\definecolor{blec}{HTML}{000000}


% Colors
\setbeamercolor*{block body}{bg=normal text.bg!93!grin}
\setbeamercolor*{block title}{bg=grin!25!uait}

\setbeamercolor{normal text}{fg=blec!65!grin,bg=uait}

\setbeamercolor{section in toc}{fg=grin}
\setbeamercolor{subsection in toc}{fg=grin!90!uait}

\setbeamercolor{frametitle}{fg=grin,bg=grin!20}

\setbeamercolor*{palette primary}{use=structure,fg=blec,bg=grin!40!uait}
\setbeamercolor*{palette secondary}{use=structure,fg=uait,bg=grin!60!uait}
\setbeamercolor*{palette tertiary}{use=structure,fg=uait,bg=grin!90!uait}
\setbeamercolor*{palette quaternary}{fg=uait,bg=blec}

\setbeamercolor*{sidebar}{use=structure,bg=grin}

\setbeamercolor*{palette sidebar primary}{use=structure,fg=grin!10}
\setbeamercolor*{palette sidebar secondary}{fg=grin!10!uait}
\setbeamercolor*{palette sidebar tertiary}{use=structure,fg=grin!50}
\setbeamercolor*{palette sidebar quaternary}{fg=grin!5!uait}

% Section balls
\setbeamerfont{section number projected}{family=\rmfamily,series=\bfseries,size=\normalsize}
\setbeamercolor{section number projected}{bg=grin!70!uait,fg=uait}

% Itemize balls
\setbeamercolor{item projected}{bg=grin!70!uait}
\setbeamercolor{subitem projected}{bg=grin!90!uait}

% Title
\setbeamercolor*{titlelike}{use=structure,fg=grin}
%lst colors

\definecolor{commentgreen}{RGB}{162, 249, 103}
\definecolor{eminence}{RGB}{108,48,130}
\definecolor{weborange}{RGB}{255,165,0}
\definecolor{frenchplum}{RGB}{254,215,0}
\definecolor{goodred}{RGB}{233, 38, 95}
\definecolor{washedyellow}{RGB}{223, 206, 94}
\definecolor{monoblue}{RGB}{79, 201, 232}

% blue code background
\lstset{
    language=c++,
    frame=single,
    framerule=0pt,
    backgroundcolor=\color{grin!70!blec},
   % basicstyle=\ttfamily\color{white},
    xleftmargin=4.3pt,
    xrightmargin=4.3pt,
    commentstyle=\color{commentgreen},
    stringstyle=\color{washedyellow},
    basicstyle=\small\ttfamily\color{white}, % basic font setting
    keywordstyle = \color{eminence},
    keywordstyle = [2]{\color{weborange}},
    keywordstyle = [3]{\color{monoblue}},
    otherkeywords = {>,<,;,.,-,!,=,~,memcpy,send_request_cgi,encoded,gsub},
    morekeywords = [2]{>,<,;,.,-,!,=,~},
    morekeywords = [3]{memcpy,send_request_cgi,encoded,gsub},
    %emph={int,char,double,float,unsigned,void,bool},
    %emphstyle={\color{blue}},
    escapechar=\&
}


% transparent background in slides
\setbeamercovered{transparent}
\setbeameroption{show notes}

\newcounter{numeroes}[section]
\newcounter{requestcounter}

\DeclareMathAlphabet{\pazocal}{OMS}{zplm}{m}{n}

%\renewenvironment{proof}{\noindent\textbf{Soluzione:}}{}
\newtheorem{exercise}{Esercizio}

\addto\captionsitalian{\renewcommand\proofname{Soluzione}}
\renewcommand*{\proofname}{Solution}
\renewcommand\qedsymbol{$\blacksquare$}

\newcommand{\mean}[1]{$\mathbb{E}[#1]$}
\newcommand{\varr}[1]{$\mathbb{V}ar[#1]$}
%\renewcommand\qedsymbol{$\square$}

\setlength{\jot}{10pt}

% tikz round photos
\newcommand{\roundpic}[4][]{
	\tikz\node [circle, minimum width = #2,
		path picture = {
				\node [#1] at (path picture bounding box.center) {
					\includegraphics[width=#3]{#4}};
			}] {};}

% Header slide
\title[Tutorato Probabilità]
{Tutorato di\\Calcolo delle Probabilità}

\author[DISI]{%
	\texorpdfstring{%
		\begin{columns}
			\column{.50\linewidth}
			\centering
			Camilla Righetti
			\column{.50\linewidth}
			\centering
			Matteo Franzil
		\end{columns}
	}{DISI}
}

\institute[UniTN]{
	\texorpdfstring{%
		\includegraphics[width=.6\textwidth]{drawable/logos/logo-disi.png}
	}{University of Trento}\\
	\smallskip
	%\hphantom{...}\\
	Corso di Laurea in Ingegneria Informatica, delle Comunicazioni ed Elettronica
}

\date[06/04/2022]{20 aprile 2022}

\begin{document}

\frame{\titlepage}

\section{Correzioni provetta 15/04/2022}

% Punteggio assegnato a ciascun esercizio
\def\pointsQa{6}
\def\pointsQb{6}
\def\pointsQc{8}
\def\pointsQd{8}
\def\pointsQe{5}

\begin{frame}[fragile]
	\frametitle{Esercizio 1}

	\begin{exercise}[\textit{Password cdp} -- \pointsQa pt]
		Il sito dedicato al calcolo delle probabilit\`a ``\texttt{cdp.com}'' richiede ai suoi utenti di registrarsi con una password. Le regole per la costruzione della password sono le seguenti:

		\begin{itemize}[-]
			\item deve essere lunga esattamente 5 caratteri;
			\item lettere maiuscole e minuscole sono considerate distinte (la password è case-sensitive);
			\item deve contenere almeno una lettera (non importa se maiuscola o minuscola) e almeno un simbolo (punto . oppure underscore \_);
			\item le lettere possibili sono quelle dell'alfabeto inglese (26 lettere);
			\item sono consentiti solamente lettere maiuscole o minuscole, il punto (.) e l'underscore (\_).
		\end{itemize}
	\end{exercise}
\end{frame}

\begin{frame}[fragile]
	\frametitle{Esercizio 1}

	\begin{exercise}[\textit{Password cdp} -- \pointsQa pt]

		\begin{enumerate}[(a)]
			\item Quante sono le password possibili?
			\item Quante delle password precedenti contengono la stringa ``\texttt{cdp}'' al loro interno? La stringa si intende scritta con caratteri maiuscoli e/o minuscoli a piacere, sono opzioni distinte sia ``\texttt{CdP}'' che ``\texttt{cDP}''.
		\end{enumerate}
	\end{exercise}
\end{frame}

\begin{frame}[fragile]
	\begin{proof}\renewcommand{\qedsymbol}{$\square$}
		a) Se le 26 lettere maiuscole sono considerate distinte dalle 26 minuscole, e aggiungendo 2 simboli grafici, ottengo un totale di $26+26+2=54$ simboli. Senza regole, posso scegliere $54^5$ password possibili; da queste vanno escluse le $52^5$ consistenti solamente di lettere (senza simboli) e le $2^5$ consistenti solo di simboli (senza lettere), arrivando così a
		$$
			n= 54^5 - 52^5- 2^5 = 78 960 960
		$$
		\qedhere
	\end{proof}
\end{frame}

\begin{frame}[fragile]
	\begin{proof}%\renewcommand{\qedsymbol}{$\longrightarrow$}
		b) Ci sono due modi per scrivere ciascuna lettera della stringa ``\texttt{cdp}'' (lettera maiuscola oppure minuscola), per un totale di $2^3=8$ modi diversi.

		\medskip

		La stringa pu\`o essere posizionata all'inizio della password, nel mezzo o alla fine (\texttt{cdp*}, \texttt{*cdp*} oppure \texttt{*cdp}), per un totale di 3 posizionamenti possibili.

		\medskip

		I rimanenti due caratteri possono essere due simboli grafici ($2^2$ possibilit\`a) oppure una lettera e un simbolo (le due lettere sono escluse, perch\'e deve essere presente almeno un simbolo, per un totale di $52\cdot2\cdot2$ modi). In totale, le password di questo tipo sono

		\begin{align*}
			m & = \underbracket{8}_{\text{modi}} \cdot \underbracket{3}_{\text{posiz.}} \cdot \: \big( \underbrace{2 \cdot 2}_{\text{simb.}} + \underbrace{52 \cdot 2 \cdot 2}_{\text{lett+simb}} \big) = 5088
		\end{align*}
	\end{proof}
\end{frame}

\begin{frame}[fragile]
	\frametitle{Esercizio 2}

	\begin{exercise}[\textit{Tombola ridotta} -- \pointsQb pt]

		Viene organizzata una tombola ``ristretta''. Vi sono 20 palline numerate (da 1 a 20) in un vaso, e ne vengono pescate (senza reinserimento) 5, una alla volta. Su ognuna delle tessere distribuite ai giocatori sono presenti 8 numeri (da 1 a 20). Due estrazioni si considerano differenti se differiscono per almeno un numero o per l'ordine di uscita dei numeri.

		\begin{enumerate}[(a)]
			\item In quanti modi diversi pu\`o avvenire l'estrazione?
			\item Sulla tessera di un giocatore sono presenti i numeri 2, 3, 5, 7, 11, 13, 17, 19. Se la prima pallina estratta è stata quella con il numero 6, per quante estrazioni differenti questa scheda realizza almeno un terno (ovvero almeno 3 dei suoi numeri vengono estratti)?
		\end{enumerate}
	\end{exercise}
\end{frame}

\begin{frame}[fragile]
	\begin{proof}%\renewcommand{\qedsymbol}{$\longrightarrow$}

		\begin{enumerate}[(a)]
			\item L'estrazione pu\`o avvenire in $D_{20,5}=\frac{20!}{15!}=1860480$ modi diversi.
			\item Visto che uno dei 5 numeri che vengono estratti non è presente sulla scheda, è possibile solo fare terno o quaterna. La quaterna può essere estratta in $$D_{8,4}=\frac{8!}{4!}=1680$$ modi. Per la terna, 3 dei numeri estratti devono essere tra gli 8 presenti sulla scheda, e l'ultimo può essere uno degli altri $20-8-1=11$ numeri. L'ultimo numero può essere estratto come secondo, terzo, quarto o quinto, per un totale di $$4\cdot11\cdot D_{8,3}=4\cdot11\cdot\frac{8!}{5!}=14784$$ modi. Sommando le due possibilità si ottiene il totale di $n=1680+14784=16464$. \qedhere
		\end{enumerate}
	\end{proof}
\end{frame}

\begin{frame}[fragile]
	\frametitle{Esercizio 3}

	\begin{exercise}[\textit{Calcio scommesse} -- \pointsQc pt]

		Un tifoso vuole scommettere sulla vittoria della propria squadra nella prossima partita di calcio. Sa che la sua squadra vince nel 70\% dei casi e pareggia nel 20\% dei casi con la squadra che la affronterà se il tempo è asciutto, mentre vince nel 30\% dei casi e pareggia nel 25\% se piove.

		\begin{enumerate}[(a)]
			\item Se le previsioni per il giorno della partita sono per il 45\% sole e per il restante pioggia, qual è la probabilità che la squadra del tifoso non perda?
			\item Sapendo che la partita è finita in pareggio, qual è la probabilità che quel giorno abbia piovuto?
		\end{enumerate}
	\end{exercise}
\end{frame}

\begin{frame}[fragile]
	\begin{proof}\renewcommand{\qedsymbol}{$\square$}

		Indichiamo con $W$, $D$, $L$ rispettivamente la vittoria (\textit{\textbf{w}in}), il pareggio (\textit{\textbf{d}raw}) e la sconfitta (\textit{\textbf{l}oss}) della squadra. Inoltre, con $R$  (\textit{\textbf{r}ain}) e $S$  (\textit{\textbf{s}un}) indichiamo la pioggia e il tempo asciutto.

		\medskip

		Piccola nota: a calcio, ``non perdere'' significa vincere oppure pareggiare.
		
		\begin{enumerate}[(a)]
			\item Affinché una squadra \textit{non perda}, calcoliamo $p(\overline{L})$: \begin{align*}
				      p(\overline{L}) & = \overbrace{p(S)\left[1 - p(L|S) \right] + p(R)\left[1 - p(L|R) \right]}^{\text{Teorema della probabilità totale}} \\
				      & = 0.45\cdot0.9+0.55\cdot0.55 \\
				      & = 0.7075
			      \end{align*}
		\end{enumerate}
	\end{proof}
\end{frame}

\begin{frame}[fragile]
	\begin{proof}%\renewcommand{\qedsymbol}{$\square$}
		\begin{enumerate}[(b)]
			\item Per come è posto il quesito (e per le richieste del primo punto), possiamo formulare la probabilità richiesta con Bayes: \begin{align*}
				      p(R|D) & = \frac{p(D|R)p(R)}{p(D)} \\
				      & = \frac{p(D|R)p(R)}{p(D|R)p(R)+p(D|S)p(S)} \\
				      & = \frac{0.25\cdot0.55}{0.25\cdot0.55+0.2\cdot0.45} \\
				      & = 0.604
			      \end{align*}
		\end{enumerate}
	\end{proof}
\end{frame}

\begin{frame}[fragile]
	\frametitle{Esercizio 4}

	\begin{exercise}[\textit{Deadline} -- \pointsQd pt]
		Uno studio di ingegneria ha recentemente vinto l'appalto per la produzione di tre progetti, A, B e C.

		\medskip

		Il progetto A non è vincolato a una scadenza e viene pagato 100. Il progetto B viene pagato 120 se lo studio lo finisce in tempo, e 80 se la deadline non viene rispettata; la deadline verrà rispettata con una probabilità del 50\%. Il terzo progetto, del valore di 90, verrà completato rispettando la scadenza prevista con la probabilità del 70\% e viene pagato esclusivamente se completato entro la deadline.

		\medskip

		Qual è la probabilità che lo studio venga pagato almeno 200?
	\end{exercise}
\end{frame}

\begin{frame}[fragile]
	\begin{proof}%\renewcommand{\qedsymbol}{$\square$}
		Siano A, B, C gli eventi che rappresentano il rispetto delle deadline per i progetti corrispondenti. Allora $p(A)=1$, $p(B)=0.5$, $p(C)=0.7$. Sia $T$ il pagamento totale ricevuto dall'azienda.

		\medskip

		Il primo progetto porta sicuramente 100 all'azienda, dunque è sufficiente che il progetto B e C combinati vengano pagati almeno 100. Se B rispetta la scadenza, si superano i 200 totali; se B non supera la scadenza, deve essere rispettata la scadenza di C.

		$$p(T\geq200)=p(B)+p(\overline{B})\cdot p(C)=0.5+0.5\cdot0.7=0.85$$
	\end{proof}
\end{frame}
\begin{frame}[fragile]
	\frametitle{Esercizio 5}

	\begin{exercise}[\textit{Crescita economica} -- \pointsQe pt]
		Una azienda prevede che, con una probabilità del 5\%, l'economia sarà in crescita. Inoltre, c'è una probabilità del 90\% che i ricavi della azienda aumentino, se l'economia sarà in crescita. Se invece l'economia non sarà in crescita, i ricavi aumenteranno solamente con una probabilità del 40\%.

		\begin{enumerate}[(a)]
			\item Qual è la probabilità che i ricavi non aumentino?
			\item Se i ricavi dopo un anno sono effettivamente aumentati, qual è la probabilità che l'economia sia stata in crescita?
		\end{enumerate}
	\end{exercise}
\end{frame}

\begin{frame}[fragile]
	\begin{proof}\renewcommand{\qedsymbol}{$\longrightarrow$}
		Indichiamo con $C$ l'evento ``economia in crescita'' e con $R$ l'aumento dei ricavi.

		Allora:

		\begin{align*}
			p(R|C) & = 0.9 \\
			p(R|\overline{C}) & = 0.4 \\
			p(C) & = 0.05 \\ \Rightarrow p(\overline{C}) & = 0.95
		\end{align*}
	\end{proof}
\end{frame}

\begin{frame}[fragile]
	\begin{proof}\renewcommand{\qedsymbol}{$\square$}
		\begin{enumerate}[(a)]
			\item Similarmente all'esercizio sul calcio, il Teorema della probabilità totale ci viene in aiuto: \begin{align*}
				      p(\overline R) = 1 - p(R) & = 1 - \big( \overbrace{p(R|C) \cdot p(C) + p(R|\overline C) \cdot p(\overline C)}^{\text{Teorema della probabilità totale}} \big) \\
				      & = 1 - (0.9 \cdot 0.05 + 0.4 \cdot 0.95 ) \\
				      & = 1 - 0.425 \\
				      & = 0.575
			      \end{align*}
		\end{enumerate}
	\end{proof}
\end{frame}

\begin{frame}[fragile]
	\begin{proof}%\renewcommand{\qedsymbol}{$\square$}
		\begin{enumerate}[(b)]
			\item Quando ci viene chiesta una probabilità con la condizione invertita, possiamo velocemente ri-invertirla: \begin{align*}
				      p(C|R) & = \frac{p(R|C)\cdot p(C)}{p(R)} \\
				      & = \frac{p(R|C) \cdot p(C)}{1 - p(\overline R)} \\ %{p(R|C)\cdot p(C)+p(R|\overline C) \cdot p(\overline C)} = \\
				      & = \frac{0.9\cdot 0.05}{0.425} \\
				      & = 0.106
			      \end{align*}
		\end{enumerate}
	\end{proof}
\end{frame}

\end{document}