\documentclass{beamer}
%\documentclass[handout]{beamer}

% This file requires a \documentclass{beamer} before and a \date{..} after

\usepackage[utf8]{inputenc}
\usepackage{graphicx}
\usepackage{ifthen}
\usepackage{moodle}
\usepackage{xcolor}
\usepackage{listings}
\usepackage{makecell}
\usepackage{tikz}
\usepackage{url}
\usepackage{mathtools}
\usepackage{ulem}
\usepackage{mathcomp}
\usepackage[italian]{babel}
\usepackage{amsmath,amsthm,amssymb,amsfonts}
\usepackage{setspace}
\usepackage{tikz}
\usepackage{soul}
\usepackage[shortlabels]{enumitem}

% Global packages
\usepackage{tikzit}

\input{style.tikzstyles}
\usetheme[secheader]{Boadilla}
\usecolortheme{dolphin}

% Main color
\definecolor{grin}{HTML}{53bbc2}
\definecolor{redd}{HTML}{b82200}
\definecolor{uait}{HTML}{ffffff}
\definecolor{blec}{HTML}{000000}


% Colors
\setbeamercolor*{block body}{bg=normal text.bg!93!grin}
\setbeamercolor*{block title}{bg=grin!25!uait}

\setbeamercolor{normal text}{fg=blec!65!grin,bg=uait}

\setbeamercolor{section in toc}{fg=grin}
\setbeamercolor{subsection in toc}{fg=grin!90!uait}

\setbeamercolor{frametitle}{fg=grin,bg=grin!20}

\setbeamercolor*{palette primary}{use=structure,fg=blec,bg=grin!40!uait}
\setbeamercolor*{palette secondary}{use=structure,fg=uait,bg=grin!60!uait}
\setbeamercolor*{palette tertiary}{use=structure,fg=uait,bg=grin!90!uait}
\setbeamercolor*{palette quaternary}{fg=uait,bg=blec}

\setbeamercolor*{sidebar}{use=structure,bg=grin}

\setbeamercolor*{palette sidebar primary}{use=structure,fg=grin!10}
\setbeamercolor*{palette sidebar secondary}{fg=grin!10!uait}
\setbeamercolor*{palette sidebar tertiary}{use=structure,fg=grin!50}
\setbeamercolor*{palette sidebar quaternary}{fg=grin!5!uait}

% Section balls
\setbeamerfont{section number projected}{family=\rmfamily,series=\bfseries,size=\normalsize}
\setbeamercolor{section number projected}{bg=grin!70!uait,fg=uait}

% Itemize balls
\setbeamercolor{item projected}{bg=grin!70!uait}
\setbeamercolor{subitem projected}{bg=grin!90!uait}

% Title
\setbeamercolor*{titlelike}{use=structure,fg=grin}
%lst colors

\definecolor{commentgreen}{RGB}{162, 249, 103}
\definecolor{eminence}{RGB}{108,48,130}
\definecolor{weborange}{RGB}{255,165,0}
\definecolor{frenchplum}{RGB}{254,215,0}
\definecolor{goodred}{RGB}{233, 38, 95}
\definecolor{washedyellow}{RGB}{223, 206, 94}
\definecolor{monoblue}{RGB}{79, 201, 232}

% blue code background
\lstset{
    language=c++,
    frame=single,
    framerule=0pt,
    backgroundcolor=\color{grin!70!blec},
   % basicstyle=\ttfamily\color{white},
    xleftmargin=4.3pt,
    xrightmargin=4.3pt,
    commentstyle=\color{commentgreen},
    stringstyle=\color{washedyellow},
    basicstyle=\small\ttfamily\color{white}, % basic font setting
    keywordstyle = \color{eminence},
    keywordstyle = [2]{\color{weborange}},
    keywordstyle = [3]{\color{monoblue}},
    otherkeywords = {>,<,;,.,-,!,=,~,memcpy,send_request_cgi,encoded,gsub},
    morekeywords = [2]{>,<,;,.,-,!,=,~},
    morekeywords = [3]{memcpy,send_request_cgi,encoded,gsub},
    %emph={int,char,double,float,unsigned,void,bool},
    %emphstyle={\color{blue}},
    escapechar=\&
}


% transparent background in slides
\setbeamercovered{transparent}
\setbeameroption{show notes}

\newcounter{numeroes}[section]
\newcounter{requestcounter}

\DeclareMathAlphabet{\pazocal}{OMS}{zplm}{m}{n}

%\renewenvironment{proof}{\noindent\textbf{Soluzione:}}{}
\newtheorem{exercise}{Esercizio}

\addto\captionsitalian{\renewcommand\proofname{Soluzione}}
\renewcommand*{\proofname}{Solution}
\renewcommand\qedsymbol{$\blacksquare$}

\newcommand{\mean}[1]{$\mathbb{E}[#1]$}
\newcommand{\varr}[1]{$\mathbb{V}ar[#1]$}
%\renewcommand\qedsymbol{$\square$}

\setlength{\jot}{10pt}

% tikz round photos
\newcommand{\roundpic}[4][]{
	\tikz\node [circle, minimum width = #2,
		path picture = {
				\node [#1] at (path picture bounding box.center) {
					\includegraphics[width=#3]{#4}};
			}] {};}

% Header slide
\title[Tutorato Probabilità]
{Tutorato di\\Calcolo delle Probabilità}

\author[DISI]{%
	\texorpdfstring{%
		\begin{columns}
			\column{.50\linewidth}
			\centering
			Camilla Righetti
			\column{.50\linewidth}
			\centering
			Matteo Franzil
		\end{columns}
	}{DISI}
}

\institute[UniTN]{
	\texorpdfstring{%
		\includegraphics[width=.6\textwidth]{drawable/logos/logo-disi.png}
	}{University of Trento}\\
	\smallskip
	%\hphantom{...}\\
	Corso di Laurea in Ingegneria Informatica, delle Comunicazioni ed Elettronica
}

\date[23/03/2022]{23 marzo 2022}

\begin{document}

\frame{\titlepage}

\section{Introduzione}

\begin{frame}[fragile]
	\frametitle{Informazioni tecniche}
	
	Quando?
	
	\begin{itemize}
	    \item ogni mercoledì, 15:30 $\Rightarrow$ 17:30
	\end{itemize}
	
	\medskip
	
	Dove?
	
	\begin{itemize}
	    \item aula A207 \\ (salvo eccezioni che vi saranno comunicate)
	\end{itemize}
	
	\medskip
	
	Chi?
	
	\begin{itemize}
	    \item Camilla Righetti \\ \verb=camilla.righetti-1@studenti.unitn.it=
	        
	    \item Matteo Franzil \\ \verb=matteo.franzil@studenti.unitn.it=
	\end{itemize}
	
\end{frame} 

\begin{frame}[fragile]
	\frametitle{Informazioni tecniche}
	
	Va bene tutto, ma cosa si fa?
	
	\begin{itemize}
	    \item supporto alla preparazione delle provette/esami
	    \item risoluzione di problemi proposti: \begin{itemize}
	        \item da voi, se avete dubbi
	        \item da noi, come approfondimento
	    \end{itemize}
	\end{itemize}
	
	\medskip
	
	Cosa NON si fa:
	
	\begin{itemize}
	    \item supporto alla parte teorica\\{ \small (scrivete alla prof. Boato, vi saprà rispondere meglio di noi) }
	    \item miracoli\\{ \small (non saremmo qui a fare tutorato altrimenti) }
	\end{itemize}
    
\end{frame} 

\begin{frame}[fragile]
	\frametitle{Informazioni tecniche}
	
	Al seguente link trovate un form Google che vi permette (anonimamente!) di inviare:
	
	\begin{itemize}
	    \item problemi agli esercizi che avete incontrato
	    \item suggerimenti di esercizi per l'incontro successivo
	\end{itemize}
	
	\begin{figure}
	    \centering
	    \includegraphics[width=0.3\textwidth]{drawable/logos/qr.png}
	\end{figure}
    
\end{frame} 

\begin{frame}[fragile]
	\frametitle{Prima provetta}
	
	Da qui alla prima provetta avremo i seguenti incontri:
	
	\begin{itemize}
	    \item oggi
	    \item 30 marzo (aula A207)
	    \item 6 aprile (\textbf{aula A205})
	    \item 13 aprile (aula A207)
	\end{itemize}
	
	\medskip
	
	Se dovesse servire, non esiteremo a fissarne di ulteriori.
	
    
\end{frame} 

\begin{frame}[fragile]
	\frametitle{Your turn!}
	
	\begin{itemize}
	    \item Come procede finora il corso?
	    \item Avete dubbi su qualche argomento in particolare?
	    \item Su cosa preferireste ci concentrassimo?
	\end{itemize}
    
\end{frame} 

\begin{frame}[fragile]
	\frametitle{Esercizio 1.1}
	
	\begin{exercise}[1.1, ex. Esercizio 11\\PDF "Esercizi con soluzioni" su Moodle]
	12 gettoni numerati vengono consegnati casualmente, uno alla volta, ad altrettante persone. Meravigliando tutti sono assegnati secondo l'ordine crescente dei numeri che li distingue. Quale è la probabilità di un simile evento?
    \end{exercise}
\end{frame}

\begin{frame}

    \begin{proof}
	La probabilità che la permutazione dei gettoni $[1, 2...12]$ avvenga è esattamente la stessa di una qualunque altra permutazione di gettoni.
	
	In altre parole, la probabilità è pari a $\frac{1}{12!}$, dove $12!$ corrisponde a tutte le permutazioni possibili dei gettoni.
	
	\medskip
	
	E' possibile vedere questa soluzione in un altro modo: quando assegno il gettone alla prima persona, ho $P = \frac{1}{12}$ di assegnarle il gettone numerato $1$. Alla seconda persona, ora ho $P =\frac{1}{11}$ di assegnarle il gettone $2$. E così via, ottenendo quindi che la probabilità finale sarà pari a
	
	\[
        \frac{1}{12} \cdot \frac{1}{11} \cdot \frac{1}{10} \cdot \cdots \cdot 1 = \frac{1}{12!}
    \]
    
    \qedhere
\end{proof}

\end{frame}

\begin{frame}[fragile]
	\frametitle{Esercizio 1.2}
	
	\begin{exercise}[1.2, ex. Esercizio 20, stesso PDF]
	Supponiamo di dover selezionare a caso 5 persone da un gruppo di 20 individui formato da 10 coppie sposate e che si sia interessati a calcolare la probabilità che i cinque individui selezionati non siano in relazione tra loro, ovvero che non ce ne siano due tra loro sposati?
    \end{exercise}
\end{frame}

\begin{frame}

    \begin{proof}\renewcommand{\qedsymbol}{$\longrightarrow$}
	Calcoliamo tutte le possibili combinazioni di 5 persone scelte da un gruppo di 20 individui: $\binom{20}{5} = 15504$.
	
	\medskip
	
	Calcoliamo ora le combinazioni a noi favorevoli. La prima persona la possiamo scegliere in 20 modi diversi. Una volta scelta la prima persona, siamo necessariamente vincolati a non scegliere il ``coniuge''.
	
	\medskip
	
	Ci rimangono quindi 18 persone da scegliere per la persona successiva, 16 per quella dopo, e così via. Otteniamo quindi $20 \cdot 18 \cdot 16 \cdot 14 \cdot 12$. Tuttavia, dobbiamo necessariamente dividere questo numero per tutte le permutazioni possibili di 5 persone, ovvero 5!.
	\end{proof}
\end{frame}

\begin{frame}

    \begin{proof}
	
	In formula:
	
	\[
		\frac{\frac{20 \cdot 18 \cdot 16 \cdot 14 \cdot 12}{5!}}{\binom{20}{5}} =  \frac{8064}{15504} = \frac{168}{323}
	\]
	
	\medskip
	P.S.: vi sono svariati modi di risolvere questo esercizio: usando le disposizioni, le combinazioni, oppure con l'approccio classico $ \frac{\text{casi favor.}}{\text{casi totali}}$ (come qua sopra).
	
	In generale, molti esercizi ammettono più ragionamenti che vanno benissimo. L'importante è capire perché si è fatta una scelta piuttosto che un'altra. Spesso, ripeterlo a parole aiuta tantissimo.
\end{proof}

\end{frame}

\begin{frame}[fragile]
    \frametitle{Un piccolo recap...}
    
    \begin{tabular}{|ccc|}
    \hline
    Operazione & Con ripetizione & Senza ripetizione \\ \hline
    \makecell{Disposizioni di\\$n$ elementi\\in $k$ posizioni} & $D_{n,k} = \frac{n!}{(n-k)!}$ & $D^r_{n,k} = n^k$ \\ 
    Permutazioni di $n$ elementi &  $P_n = n!$ & $P^r_{n,k} = n^n$ \\
    \makecell{Combinazioni di n elementi\\presi k alla volta} & $C_{n,k} = \binom{n}{k} = \frac{n!}{k!(n-k)!}$ & $C^r_{n,k} = \frac{(n+k-1)!}{k!(n-1)!}$ \\ \hline
    \end{tabular}

    
\end{frame}

%\frame{\titlepage}

\end{document}
