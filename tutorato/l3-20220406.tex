% !TEX root = .

\documentclass{beamer}
%\documentclass[handout]{beamer}

\usepackage[utf8]{inputenc}
\usepackage{graphicx}
\usepackage{ifthen}
\usepackage{xcolor}
\usepackage{listings}
\usepackage{makecell}
\usepackage{tikz}
\usepackage{url}
\usepackage{mathtools}
\usepackage{ulem}
\usepackage{mathcomp}
\usepackage[italian]{babel}
\usepackage{amsmath,amsthm,amssymb,amsfonts}
\usepackage{setspace}
\usepackage{tikz}
\usepackage{soul}

% Global packagesl
\usepackage{tikzit}
\input{style.tikzstyles}

\usetheme[secheader]{Boadilla}
\usecolortheme{dolphin}

% Main color
\definecolor{grin}{HTML}{53bbc2}
\definecolor{redd}{HTML}{b82200}
\definecolor{uait}{HTML}{ffffff}
\definecolor{blec}{HTML}{000000}


% Colors
\setbeamercolor*{block body}{bg=normal text.bg!93!grin}
\setbeamercolor*{block title}{bg=grin!25!uait}

\setbeamercolor{normal text}{fg=blec!65!grin,bg=uait}

\setbeamercolor{section in toc}{fg=grin}
\setbeamercolor{subsection in toc}{fg=grin!90!uait}

\setbeamercolor{frametitle}{fg=grin,bg=grin!20}

\setbeamercolor*{palette primary}{use=structure,fg=blec,bg=grin!40!uait}
\setbeamercolor*{palette secondary}{use=structure,fg=uait,bg=grin!60!uait}
\setbeamercolor*{palette tertiary}{use=structure,fg=uait,bg=grin!90!uait}
\setbeamercolor*{palette quaternary}{fg=uait,bg=blec}

\setbeamercolor*{sidebar}{use=structure,bg=grin}

\setbeamercolor*{palette sidebar primary}{use=structure,fg=grin!10}
\setbeamercolor*{palette sidebar secondary}{fg=grin!10!uait}
\setbeamercolor*{palette sidebar tertiary}{use=structure,fg=grin!50}
\setbeamercolor*{palette sidebar quaternary}{fg=grin!5!uait}

% Section balls
\setbeamerfont{section number projected}{family=\rmfamily,series=\bfseries,size=\normalsize}
\setbeamercolor{section number projected}{bg=grin!70!uait,fg=uait}

% Itemize balls
\setbeamercolor{item projected}{bg=grin!70!uait}
\setbeamercolor{subitem projected}{bg=grin!90!uait}

% Title
\setbeamercolor*{titlelike}{use=structure,fg=grin}
%lst colors

\definecolor{commentgreen}{RGB}{162, 249, 103}
\definecolor{eminence}{RGB}{108,48,130}
\definecolor{weborange}{RGB}{255,165,0}
\definecolor{frenchplum}{RGB}{254,215,0}
\definecolor{goodred}{RGB}{233, 38, 95}
\definecolor{washedyellow}{RGB}{223, 206, 94}
\definecolor{monoblue}{RGB}{79, 201, 232}

% blue code background
\lstset{
    language=c++,
    frame=single,
    framerule=0pt,
    backgroundcolor=\color{grin!70!blec},
   % basicstyle=\ttfamily\color{white},
    xleftmargin=4.3pt,
    xrightmargin=4.3pt,
    commentstyle=\color{commentgreen},
    stringstyle=\color{washedyellow},
    basicstyle=\small\ttfamily\color{white}, % basic font setting
    keywordstyle = \color{eminence},
    keywordstyle = [2]{\color{weborange}},
    keywordstyle = [3]{\color{monoblue}},
    otherkeywords = {>,<,;,.,-,!,=,~,memcpy,send_request_cgi,encoded,gsub},
    morekeywords = [2]{>,<,;,.,-,!,=,~},
    morekeywords = [3]{memcpy,send_request_cgi,encoded,gsub},
    %emph={int,char,double,float,unsigned,void,bool},
    %emphstyle={\color{blue}},
    escapechar=\&
}


% transparent bg
\setbeamercovered{transparent}
\setbeameroption{show notes}


\newcounter{numeroes}[section]
\newcounter{requestcounter}

\DeclareUnicodeCharacter{2212}{\textminus}

\DeclareMathAlphabet{\pazocal}{OMS}{zplm}{m}{n}

%\renewenvironment{proof}{\noindent\textbf{Soluzione:}}{}
\newtheorem{exercise}{Esercizio}
\renewcommand*{\proofname}{Soluzione}
\renewcommand\qedsymbol{$\blacksquare$}

\newcommand{\mean}[1]{$\mathbb{E}[#1]$}
\newcommand{\varr}[1]{$\mathbb{V}ar[#1]$}
%\renewcommand\qedsymbol{$\square$}


% tikz round photos
\newcommand{\roundpic}[4][]{
	\tikz\node [circle, minimum width = #2,
		path picture = {
				\node [#1] at (path picture bounding box.center) {
					\includegraphics[width=#3]{#4}};
			}] {};}

% Header slide
\title[Tutorato Probabilità]
{Tutorato di\\Calcolo delle Probabilità}

\author[DISI]{%
	\texorpdfstring{%
		\begin{columns}
			\column{.50\linewidth}
			\centering
			Camilla Righetti
			\column{.50\linewidth}
			\centering
			Matteo Franzil
		\end{columns}
	}{DISI}
}

\institute[UniTN]{
	\texorpdfstring{%
		\includegraphics[width=.6\textwidth]{drawable/logos/logo-disi.png}
	}{University of Trento}\\
	\smallskip
	%\hphantom{...}\\
	Corso di Laurea in Ingegneria Informatica, delle Comunicazioni ed Elettronica
}

\date[06/04/2022]{6 aprile 2022}

\begin{document}

\frame{\titlepage}

\section{Introduzione}

\begin{frame}
	\frametitle{Siamo in A205...}

	...solo oggi.

	\medskip

	La prossima volta in A207.

	\medskip

	Non è colpa nostra se l'orario non lo aggiornano.

\end{frame}

\begin{frame}[fragile]
	\frametitle{Un piccolo recap...}

	\begin{center}
		\begin{tabular}{|ccc|}
			\hline
			Operazione                   & Con ripetizione & Senza ripetizione \\ \hline
			\makecell{Disposizioni di                                          \\$n$ elementi\\in $k$ posizioni} & $D_{n,k} = \frac{n!}{(n-k)!}$ & $D^r_{n,k} = n^k$ \\
			Permutazioni di $n$ elementi & $P_n = n!$      & $P^r_{n,k} = n^n$ \\
			\makecell{Combinazioni di n elementi                               \\presi k alla volta} & $C_{n,k} = \binom{n}{k} = \frac{n!}{k!(n-k)!}$ & $C^r_{n,k} = \frac{(n+k-1)!}{k!(n-1)!}$ \\ \hline
		\end{tabular}
	\end{center}

\end{frame}

\begin{frame}[fragile]
	\frametitle{...anche in altre parole}

	\begin{center}

		\begin{tabular}{|ccc|}
			\hline
			               & modalità                           & di estrazione \\ \hline
			configurazione & senza rimessa                      & con rimessa   \\ \hline
			ordinata       & $D_{n, k}$ = $\frac{P_n}{(n-k)!}$  & $D^r_{n, k}$  \\
			non ordinata   & $C_{n, k}$ = $\frac{D_{n, k}}{k!}$ & $C^r_{n_k}$   \\ \hline
		\end{tabular}
	\end{center}

\end{frame}

\begin{frame}[fragile]
	\frametitle{Esercizio 3.1}

	\begin{exercise}[3.1, ex. Esercizio 12\\PDF "Esercizi" su Moodle]
		Quattro ragazzi e quattro ragazze partecipano ad una caccia al tesoro a coppie.
		\begin{enumerate}
			\item Se ciascuna coppia è formata da un ragazzo e da una ragazza in quanti modi diversi si può formare la quaterna di coppie?
			\item Se ciascuna coppia è formata da persone dello stesso sesso, in quanti modi diversi si può formare la quaterna di coppie?
			\item In quanti modi diversi si può formare la quaterna di coppie (insieme di due elementi)?
		\end{enumerate}
	\end{exercise}
\end{frame}

\begin{frame}[fragile]
	\begin{proof}[Risoluzione]
		\begin{enumerate}
			\item $\frac{(4 \cdot 4) \cdot (3 \cdot 3) \cdot (2 \cdot 2) \cdot (1 \cdot 1)}{4!} = \frac{16 \cdot 9 \cdot 4 \cdot 1}{24} = 4!$. Per la prima coppia, possiamo scegliere 4 ragazzi e 4 ragazze. Per la seconda, 3 e 3, e così via. Dividiamo tutto per 4!, ovvero le permutazioni possibili -- in quanto non ci interessa l'ordine e una disposizione $A_1A_2, B_1B_2, \cdots$ è sempre uguale a $B_1B_2,  A_1A_2, \cdots$.
			\item $ (\binom{4}{2} \cdot \frac{1}{2!}) \cdot (\binom{4}{2} \cdot \frac{1}{2!})$. Per i maschi, scegliamo dai 4 disponibili 2 persone (l'altra coppia è di conseguenza vincolata). Dobbiamo contare il caso in cui le coppie sono scambiate. Stessa cosa per le femmine.
			\item Stavolta non abbiamo vincoli legati al genere dei partecipanti delle coppie, quindi possiamo trattare il problema come una permutazione con ripetizione, ovvero: $\frac{8!}{2!2!2!2!} \cdot \frac{1}{4!}$. In altre parole, abbiamo le permutazioni di tutti gli 8 partecipanti, divise per le permutazioni delle singole coppie (in quanto $AB = BA$ per una coppia), diviso infine per le permutazioni di tutte le coppie (in quanto $1234$ e $1423$ sono equivalenti). \qedhere
		\end{enumerate}
		\qedhere
	\end{proof}

\end{frame}

\begin{frame}[fragile]
	\frametitle{Esercizio 3.2}
	\begin{exercise}[3.2, esame 28/06/2016 degli Informatici]
		Un messaggio di posta elettronica viene spedito dal calcolatore A al calcolatore B. Il collegamento tra i due calcolatori è di questo tipo, ogni messaggio in partenza da A viene spedito a C e D, i quali spediscono il messaggio a B.

		Ad ogni passaggio il messaggio può arrivare integro o non integro (il controllo viene fatto attraverso un codice), nel caso in cui il messaggio è integro viene re-inviato secondo lo schema descritto sopra, altrimenti il messaggio viene cancellato e perduto.
	\end{exercise}
\end{frame}

\begin{frame}[fragile]
	\frametitle{Esercizio 3.2}
	\begin{exercise}[3.2, esame 28/06/2016 degli Informatici]
		Le probabilità che da un passaggio ad un altro il messaggio arrivi integro è così descritto ($P(A \rightarrow C)$ indica la probabilità che il messaggio inviato da A giunga integro in C e così via):

		\begin{gather*}
			P(A \rightarrow C) = 0.79, P(A \rightarrow D) = 0.92 \\
			P(C \rightarrow B|A \rightarrow C) = 0.81, P(D \rightarrow B|A \rightarrow D) = 0.66 \\
			P(C \rightarrow B|\overline{A \rightarrow C}) = 0, P(D \rightarrow B|\overline{A \rightarrow D}) = 0
		\end{gather*}
	\end{exercise}
\end{frame}

\begin{frame}[fragile]
	\frametitle{Esercizio 3.2}
	\begin{exercise}[3.2, esame 28/06/2016 degli Informatici]
		Infine, tutti gli eventi che riguardano una linea di trasmissione ($A \rightarrow C \rightarrow B$) sono stocasticamente indipendenti da quelli dell'altra linea ($A \rightarrow D \rightarrow B$).
		\begin{enumerate}
			\item Qual è la probabilità che il messaggio venga recapitato SOLO attraverso il canale $A \rightarrow C \rightarrow B$?
			\item Qual è la probabilità che il messaggio venga recapitato attraverso il canale $A \rightarrow C \rightarrow B$?
			\item Qual è la probabilità che il messaggio venga recapitato?
			\item Sapendo che il messaggio è stato ricevuto da B qual è la probabilità dell'evento $A \rightarrow D$?
		\end{enumerate}
	\end{exercise}
\end{frame}

\renewcommand\qedsymbol{$\square$}

\begin{frame}
	\begin{proof}[Punto 1.]\renewcommand{\qedsymbol}{$\longrightarrow$}
		\begin{center}
			\begin{tikzpicture}
				\begin{pgfonlayer}{nodelayer}
					\node [style=new style 0] (0) at (0, 0) {A};
					\node [style=new style 0] (1) at (6, 0) {B};
					\node [style=new style 0] (3) at (3, -3) {D};
					\node [style=new style 0] (4) at (3, 3) {C};
				\end{pgfonlayer}
				\begin{pgfonlayer}{edgelayer}
					\draw [style=edge, bend left=45] (0) to (4);
					\draw [style=edge, bend right=45] (0) to (3);
					\draw [style=edge, bend right=45] (3) to (1);
					\draw [style=edge, bend left=45] (4) to (1);
				\end{pgfonlayer}
			\end{tikzpicture}
		\end{center} \qedhere
	\end{proof}
\end{frame}

\begin{frame}
	\begin{proof}[Punto 1.]\renewcommand{\qedsymbol}{$\longrightarrow$}
		Affinché il pacchetto venga recapitato soltanto attraverso il percorso $A \rightarrow C \rightarrow B$, dobbiamo imporre che il pacchetto \textbf{non} passi dall'altro, contemporaneamente:

		\[
			P((A \rightarrow C \rightarrow B) \cap \overline{(A \rightarrow D \rightarrow B)}) = P(A \rightarrow C \rightarrow B)P(\overline{A \rightarrow D \rightarrow B})
		\]

		Grazie all'indipendenza dei due eventi, siamo in grado di trasformare l'intersezione degli eventi nella rispettiva moltiplicazione. Ora possiamo trasformare separatamente i due casi nelle relative probabilità condizionate, utilizzando il teorema di Bayes.

		\begin{equation*}
			\begin{split}
				&P(A \rightarrow C \rightarrow B) \cdot P(\overline{A \rightarrow D \rightarrow B}) = \\
				&(P(C \rightarrow B|A \rightarrow C)P(A \rightarrow C)) \cdot P(\overline{A \rightarrow D \rightarrow B})
			\end{split}
		\end{equation*}
		\qedhere
	\end{proof}
\end{frame}
\begin{frame}
	\begin{proof}[Punto 1.]

		Il primo caso è abbastanza semplice: è sufficiente utilizzare la formula inversa. Per il secondo caso, invece, dobbiamo dividere la formula: infatti,

		\[
			P(\overline{A \rightarrow D \rightarrow B}) =
		\]
		\[
			P(\overline{D \rightarrow B}|A \rightarrow D)P(A \rightarrow D) + P(\overline{D \rightarrow B}|\overline{A \rightarrow D})P(\overline{A \rightarrow D})
		\]

		in quanto dobbiamo tenere conto sia dell'eventualità che il pacchetto venga perso nel tratto $AD$, sia nel tratto $DB$. Questo porta a dover dividere la probabilità in due e a considerarle separatamente come somma.

		\begin{gather*}
			\left( P(C \rightarrow B|A \rightarrow C)P(A \rightarrow C) \right) \cdot \\
			\left( P(\overline{D \rightarrow B}|A \rightarrow D)P(A \rightarrow D) + P(\overline{D \rightarrow B}|\overline{A \rightarrow D})P(\overline{A \rightarrow D}) \right) = \\
			= 0.251
		\end{gather*} \qedhere
	\end{proof}
\end{frame}


\begin{frame}
	\begin{proof}[Punto 2.]
		Questo punto lo si ricava direttamente dal precedente: infatti, richiedere la probabilità che il messaggio venga recapitato (senza solo) dal primo canale è pari a $P(A \rightarrow C \rightarrow B)$, che possiamo velocemente espandere in $P(C \rightarrow B|A \rightarrow C)P(A \rightarrow C) = 0.6399$ \qedhere
	\end{proof}
\end{frame}


\begin{frame}
	\begin{proof}[Punto 3.]\renewcommand{\qedsymbol}{$\longrightarrow$}
		In questo caso, invece, ci viene richiesta la probabilità che il messaggio venga recapitato e basta. Allora si tratta di un unione di eventi.

		Tuttavia, per risolvere un unione, non possiamo affidarci alle semplici moltiplicazioni:

		\[
			P((A \rightarrow C \rightarrow B) \cup \overline{(A \rightarrow D \rightarrow B)}) \ne P(A \rightarrow C \rightarrow B)P(\overline{A \rightarrow D \rightarrow B})
		\] \qedhere
	\end{proof}
\end{frame}

\begin{frame}
	\begin{proof}[Punto 3.]
		Fortunatamente, le nozioni insiemistiche ci soccorrono: infatti, vale $P(A \cup B) = P(A) + P(B) - P(A \cap B)$.

		Allora procediamo con la sostituzione:

		\only<2>{
			\renewcommand{\qedsymbol}{$\longrightarrow$}
			\begin{equation*}
				\begin{split}
					P((A \rightarrow C \rightarrow B) \cup (A \rightarrow D \rightarrow B)) & = \qquad\text{\footnotesize (a)}
				\end{split}
			\end{equation*}
		}

		\only<3>{
			\renewcommand{\qedsymbol}{$\longrightarrow$}
			\begin{equation*}
				\begin{split}
					P(A \rightarrow C \rightarrow B) & + \\
					P(A \rightarrow D \rightarrow B) & - \\
					P((A \rightarrow C \rightarrow B) \cap (A \rightarrow D \rightarrow B)) & = \qquad\text{\footnotesize (b)}
				\end{split}
			\end{equation*}
		}

		\only<4>{
			\renewcommand{\qedsymbol}{$\longrightarrow$}
			\begin{equation*}
				\begin{split}
					P(C \rightarrow B|A \rightarrow C)P(A \rightarrow C) & + \\
					(D \rightarrow B|A \rightarrow D)P(A \rightarrow D) & - \\
					P((A \rightarrow C \rightarrow B) \cap (A \rightarrow D \rightarrow B)) & =  \qquad\text{\footnotesize (c)}
				\end{split}
			\end{equation*}
		}

		\only<5>{
			\renewcommand{\qedsymbol}{$\longrightarrow$}
			\begin{equation*}
				\begin{split}
					\underbrace{0.81 \cdot 0.79}_{P(A \rightarrow C \rightarrow B)} +
					\underbrace{0.66 \cdot 0.92}_{P(A \rightarrow D \rightarrow B)} -
					\underbrace{(0.81 \cdot 0.79) \cdot (0.66 \cdot 0.92)}_{P((A \rightarrow C \rightarrow B) \cap (A \rightarrow D \rightarrow B)) } & = 0.859
				\end{split}
			\end{equation*}
		}

		\only<6>{
			% Equazione completa
			\begin{equation*}
				\begin{split}
					P((A \rightarrow C \rightarrow B) \cup (A \rightarrow D \rightarrow B)) & = \qquad\text{\footnotesize (a)} \\
					P(A \rightarrow C \rightarrow B) & + \\
					P(A \rightarrow D \rightarrow B) & - \\
					P((A \rightarrow C \rightarrow B) \cap (A \rightarrow D \rightarrow B)) & = \qquad\text{\footnotesize (b)}  \\
					P(C \rightarrow B|A \rightarrow C)P(A \rightarrow C) & + \\
					(D \rightarrow B|A \rightarrow D)P(A \rightarrow D) & - \\
					P((A \rightarrow C \rightarrow B) \cap (A \rightarrow D \rightarrow B)) & = \qquad\text{\footnotesize (c)} \\
					\underbrace{0.81 \cdot 0.79}_{P(A \rightarrow C \rightarrow B)} +
					\underbrace{0.66 \cdot 0.92}_{P(A \rightarrow D \rightarrow B)} -
					\underbrace{(0.81 \cdot 0.79) \cdot (0.66 \cdot 0.92)}_{P((A \rightarrow C \rightarrow B) \cap (A \rightarrow D \rightarrow B)) } & = 0.859 \qedhere
				\end{split}
			\end{equation*}
		}
	\end{proof}
\end{frame}
\renewcommand\qedsymbol{$\blacksquare$}


\begin{frame}
	\begin{proof}[Punto 4.]
		Questo punto presenta una richiesta abbastanza schietta: calcolare $P(A \rightarrow D|A \rightarrow B)$. Con il teorema di Bayes, la liquidiamo velocemente:

		\[
			P(A \rightarrow D|A \rightarrow B) = \frac{P((A \rightarrow D) \cap (A \rightarrow B))}{P(A \rightarrow B)} = \frac{P(A \rightarrow D \rightarrow B)}{P(A \rightarrow B)}
		\]

		dove l'ultimo passaggio è giustificato dal fatto che se dobbiamo cercare l'intersezione tra la probabilità che un messaggio venga consegnato e che passi per B, siamo necessariamente "costretti" a cercare la probabilità che sia passato per quel canale. I calcoli numerici seguono per sostituzione. \qedhere
	\end{proof}
\end{frame}

\end{document}