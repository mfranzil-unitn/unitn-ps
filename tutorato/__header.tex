% This file requires a \documentclass{beamer} before and a \date{..} after

\usepackage[utf8]{inputenc}
\usepackage{graphicx}
\usepackage{ifthen}
\usepackage{moodle}
\usepackage{xcolor}
\usepackage{listings}
\usepackage{makecell}
\usepackage{tikz}
\usepackage{url}
\usepackage{mathtools}
\usepackage{ulem}
\usepackage{mathcomp}
\usepackage[italian]{babel}
\usepackage{amsmath,amsthm,amssymb,amsfonts}
\usepackage{setspace}
\usepackage{tikz}
\usepackage{soul}
\usepackage[shortlabels]{enumitem}

% Global packages
\usepackage{tikzit}

\input{style.tikzstyles}
\usetheme[secheader]{Boadilla}
\usecolortheme{dolphin}

% Main color
\definecolor{grin}{HTML}{53bbc2}
\definecolor{redd}{HTML}{b82200}
\definecolor{uait}{HTML}{ffffff}
\definecolor{blec}{HTML}{000000}


% Colors
\setbeamercolor*{block body}{bg=normal text.bg!93!grin}
\setbeamercolor*{block title}{bg=grin!25!uait}

\setbeamercolor{normal text}{fg=blec!65!grin,bg=uait}

\setbeamercolor{section in toc}{fg=grin}
\setbeamercolor{subsection in toc}{fg=grin!90!uait}

\setbeamercolor{frametitle}{fg=grin,bg=grin!20}

\setbeamercolor*{palette primary}{use=structure,fg=blec,bg=grin!40!uait}
\setbeamercolor*{palette secondary}{use=structure,fg=uait,bg=grin!60!uait}
\setbeamercolor*{palette tertiary}{use=structure,fg=uait,bg=grin!90!uait}
\setbeamercolor*{palette quaternary}{fg=uait,bg=blec}

\setbeamercolor*{sidebar}{use=structure,bg=grin}

\setbeamercolor*{palette sidebar primary}{use=structure,fg=grin!10}
\setbeamercolor*{palette sidebar secondary}{fg=grin!10!uait}
\setbeamercolor*{palette sidebar tertiary}{use=structure,fg=grin!50}
\setbeamercolor*{palette sidebar quaternary}{fg=grin!5!uait}

% Section balls
\setbeamerfont{section number projected}{family=\rmfamily,series=\bfseries,size=\normalsize}
\setbeamercolor{section number projected}{bg=grin!70!uait,fg=uait}

% Itemize balls
\setbeamercolor{item projected}{bg=grin!70!uait}
\setbeamercolor{subitem projected}{bg=grin!90!uait}

% Title
\setbeamercolor*{titlelike}{use=structure,fg=grin}
%lst colors

\definecolor{commentgreen}{RGB}{162, 249, 103}
\definecolor{eminence}{RGB}{108,48,130}
\definecolor{weborange}{RGB}{255,165,0}
\definecolor{frenchplum}{RGB}{254,215,0}
\definecolor{goodred}{RGB}{233, 38, 95}
\definecolor{washedyellow}{RGB}{223, 206, 94}
\definecolor{monoblue}{RGB}{79, 201, 232}

% blue code background
\lstset{
    language=c++,
    frame=single,
    framerule=0pt,
    backgroundcolor=\color{grin!70!blec},
   % basicstyle=\ttfamily\color{white},
    xleftmargin=4.3pt,
    xrightmargin=4.3pt,
    commentstyle=\color{commentgreen},
    stringstyle=\color{washedyellow},
    basicstyle=\small\ttfamily\color{white}, % basic font setting
    keywordstyle = \color{eminence},
    keywordstyle = [2]{\color{weborange}},
    keywordstyle = [3]{\color{monoblue}},
    otherkeywords = {>,<,;,.,-,!,=,~,memcpy,send_request_cgi,encoded,gsub},
    morekeywords = [2]{>,<,;,.,-,!,=,~},
    morekeywords = [3]{memcpy,send_request_cgi,encoded,gsub},
    %emph={int,char,double,float,unsigned,void,bool},
    %emphstyle={\color{blue}},
    escapechar=\&
}


% transparent background in slides
\setbeamercovered{transparent}
\setbeameroption{show notes}

\newcounter{numeroes}[section]
\newcounter{requestcounter}

\DeclareMathAlphabet{\pazocal}{OMS}{zplm}{m}{n}

%\renewenvironment{proof}{\noindent\textbf{Soluzione:}}{}
\newtheorem{exercise}{Esercizio}

\addto\captionsitalian{\renewcommand\proofname{Soluzione}}
\renewcommand*{\proofname}{Solution}
\renewcommand\qedsymbol{$\blacksquare$}

\newcommand{\mean}[1]{$\mathbb{E}[#1]$}
\newcommand{\varr}[1]{$\mathbb{V}ar[#1]$}
%\renewcommand\qedsymbol{$\square$}

\setlength{\jot}{10pt}

% tikz round photos
\newcommand{\roundpic}[4][]{
	\tikz\node [circle, minimum width = #2,
		path picture = {
				\node [#1] at (path picture bounding box.center) {
					\includegraphics[width=#3]{#4}};
			}] {};}

% Header slide
\title[Tutorato Probabilità]
{Tutorato di\\Calcolo delle Probabilità}

\author[DISI]{%
	\texorpdfstring{%
		\begin{columns}
			\column{.50\linewidth}
			\centering
			Camilla Righetti
			\column{.50\linewidth}
			\centering
			Matteo Franzil
		\end{columns}
	}{DISI}
}

\institute[UniTN]{
	\texorpdfstring{%
		\includegraphics[width=.6\textwidth]{drawable/logos/logo-disi.png}
	}{University of Trento}\\
	\smallskip
	%\hphantom{...}\\
	Corso di Laurea in Ingegneria Informatica, delle Comunicazioni ed Elettronica
}